Proportional, Integral, Derivative (PID) Controller

$TAC_{y+1} = min\left \{ max\left \{ exp(u^1_y), exp(u^2_y)\right \}, exp(u^3_y)\right \}TAC_y$

where

$u_y = K_Pe_y + K_I \sum_{z=y-\delta}^{y}e_z + K_D(e_y-e_{y-1})$

and $u^1_y$ is the control signal and  $u^2_y$ and $u^3_y$ are lower and upper bounds repectively, which could be an interannual bound on the relative increase, or set to an upper or lower limit reference point.

$u_y$ is the control signal in year $y$ that is used for the TAC adjustment,  in control theory this type of system is known as a proportional, integral, derivative (PID) controller. The control signal is calculated from $e_y$, giving the divergence of an index relative to a reference point. The estimate can be calculated either directly from a survey or through an assessment. The desired closed-loop behaviour is then obtained by tuning the three parameters $K_P$, $K_I$ and $K_D$ where $\delta$ denotes the historical time period used to calculate the integrated (I) part of the control signal $e_y$.  The three control parameters of the HCR ($K_P$, $K_I$ and $K_D$) can be tuned. Using a historical period $\delta$  means that “moving targets” are considered, since divergence is always relative to the index in a previous year. 

<!-- A common method, named after Ziegler and Nichols and often applied in industrial engineering, is to first set $K_I$ and $K_D$ to zero and increase $K_P$ until the output of the control loop starts to oscillate with constant amplitude. The value of $K_P$ at this point is called the critical gain and can be combined with the value of the oscillation period to arrive at standard tunings for $K_P$, $K_I$ and/or $K_D$, depending on whether strictly proportional (P), proportional-integral (PI), or proportional-integral-derivative (PID) control is desirable. Here, $K_P$ represents the proportional response given the control signal, $K_I$ is the integral part of the response given the aggregation (sum) of recent control signals, and $K_D$ is the derivative part of the response driven by the rate at which the control signal varies.  -->


The control signal, u,  is the sum of three terms: a proportional term (P) that is proportional to the error, an integral term (I) that is proportional to the integral of the error, and a derivative term (D) that is proportional to the derivative of the error. 

The controller parameters are proportional gain $k_p$ , integral gain $k_i$ and derivative gain $k_d$ i.e.

\begin{equation}
 u(t) = k_pe(t)+k_i\int_0^t e(\tau)d\tau+k_d\frac{de}{dt}
\end{equation}

The controller can also be parameterised as

\begin{equation}
u(t) = k_c \left ( e(t)+\frac{1}{\tau_I}dt\int_0^t e(\tau)d\tau+T_d\frac{dt}{dt} \right)
\end{equation}

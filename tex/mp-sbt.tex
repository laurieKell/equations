CCSBT developed an MP where The TAC is an average of candidate TACs obtained from two HCRs \citep{hillary2013sbthcr}.

The first HCR used a single index for the adult stock and then increased or decreased the current catch if that index was increasing or decreasing respectively, while the second compared the current value of an index to a reference period.

In the first, the TAC is updated depending on the trend in an index ($I$)

\begin{equation}
 TAC^1_{y+1}=TAC_y\times \left\{\begin{array}{rcl}  {1-k_1|\lambda|^{\gamma}} & \mbox{for} & \lambda<0\\[0.35cm]
{1+k_2\lambda} & \mbox{for} & \lambda\geq 0 
    \end{array}\right.
\end{equation}
        

where $\lambda$ is the slope in the regression of $\ln I_y$ against year for the most recent $n$ years. $k_1$ and $k_2$ are \textit{gain} parameters and $\gamma$ actions asymmetry so that decreases in the index do not result in the same relative change as as an increase.

%giving 4 tunable parameters (\textbf{Table} \ref{tab1})

The second HCR uses both an adult and juvenile indies i.e.

\begin{equation} 
 %\begin{align*}
     TAC^2_{y+1} = 0.5\times\left(TAC_y+C^{\rm targ}_y\Delta^R_y\right)\\
 %\end{align*}
\end{equation}

where 

\begin{equation} 
 %\begin{align*}
  C^{\rm targ}_y = 
  \left\{\begin{array}{rcl} {\delta \left[\frac{I_{y}}{I^*}\right]^{1-\veps_b}} & \mbox{for} & I_{y}\geq I^*\\
                            {\delta \left[\frac{I_{y}}{I^*}\right]^{1+\veps_b}} & \mbox{for} & I_{y}<I^* \\
    \end{array}\right.
 %\end{align*}
\end{equation}

\begin{equation} 
\Delta^R_y = \left\{\begin{array}{rcl}{\left[\frac{\bar{R}}{\mathcal{R}}\right]^{1-\veps_r}} & \mbox{for} & \bar{R}\geq\mathcal{R}\\[0.35cm]
{\left[\frac{\bar{R}}{\mathcal{R}}\right]^{1+\veps_r}} & \mbox{for} & \bar{R}<\mathcal{R}
\end{array}\right.
\end{equation}


where $\delta$ is the \textit{target} catch; $I^*$ the \textit{target} adult index (e.g. a mean observed CPUE corresponding to a period where the stock was at a desired fraction of $B_0$ or $M_{MSY}$) and 
$\bar{R}$ is the average recent juvenile biomass i.e.

\begin{equation}
    \bar{R}=\frac{1}{\tau_R}\sum\limits_{i=y-\tau_R+1}^{y}R_i
 \end{equation}

 
$\mathcal{R}$ is a ``limit'' level derived from the mean recruitment over a reference period; while $\veps[0,1]$ actions asymmetry so that increases in TAC do not occur at the same level as decreases.

%There are therefore 5 tunable parameters, \textbf{Table} \ref{tab2}


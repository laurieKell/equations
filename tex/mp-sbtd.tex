A derivative control rule (D) is so called as the control signal is derived from the trend in the signal, i.e. to the derivative of the error. 


\begin{equation}
 TAC^1_{y+1}=TAC_y\times 
 \left\{\begin{array}{rcl}  
    {1-k_1|\lambda|^{\gamma}} & \mbox{for} & \lambda<0\\[0.35cm]
    {1+k_2\lambda} & \mbox{for} & \lambda\geq 0 
 \end{array}\right.
\end{equation}

where $\lambda$ is the slope in the regression of $\ln I_y$ against year for the most recent $n$ years and $k_1$ and $k_2$ are \textit{gain} parameters and $\gamma$ actions asymmetry so that decreases in the index do not result in the same relative change as as an increase.

The TAC is then the average of the last TAC and the value output by the HCR. 

\begin{equation} 
     TAC_{y+1} = 0.5\times\left(TAC_y+C^{\rm targ}_y\right)\\
\end{equation}


